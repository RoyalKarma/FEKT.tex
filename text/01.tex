\section{LaTeX rychlokurz}
\subsection{\LaTeX}
LaTeX je sázecí systém pro tvorbu odborných dokumentů. Uživatel píše plain text, který je pak dle použitých značek převeden na výsledný dokument. Oproti tomu například ve Wordu píšmene formátovaný text, jinak také \enquote{What You See Is What You Get}. Hlavními výhodami LaTeXu jsou především vysoká typografická kvalita, jednoduchá přenositelnost (už žádné rozbité word dokumenty) a jednotnost stylu (velikosti písma, typy fontů\dots - vše je definováno v šabloně).

\subsection{LaTeX na FEKTu}
Všechny oficiální šablony pro studium na FEKTu můžete najít na 
\url{https://latex.feec.vutbr.cz/}

\subsection{Jak na Overleaf?}
Nejdříve si z \url{https://github.com/VUT-FEKT-IBE/FEKT.tex}
stáhneme tuto šablonu. Dále si na overleafu vybereme možnost \texttt{New Project > Upload Project} a vybereme staženou šablonu.

Po otevření šablony najedeme na menu v levém horním rohu v možností \texttt{Main Document} a vybereme \texttt{main.tex} nebo \texttt{text/01.tex}. Dokument samotný pak kompilujeme \textbf{mimo} \texttt{shared.tex}. Jinak se PDF nevytvoří.

Pokud chcete oddělit jednotlivé textové soubory a vkládat je do dokumentu zvlášť, stačí v záložce \texttt{text} vytvořit nový soubor s příponou \texttt{.tex}. V souboru \texttt{main.tex} pak daný soubor načteme pomocí příkazu \verb|\include{file_path}|.

Dokument samotný kompilujeme pomocí tlačítka \texttt{Recompile}, případně pomocí zkratky CTRL+ENTER. Tlačítko pro stažení PDF souboru se nachází vedle \texttt{Recompile}.


\subsection{Struktura}
\begin{itemize}
    \item Pro vytvoření\uv{hlavního nadpisu} (1.; 2. \dots) použijeme příkaz \verb|\section{text}|, jednotlivé oddíly se číslují automaticky vzesupně. 
    \item Pro vytvoření podnadipsů (1.1; 2.2 \dots) použijeme příkaz  \verb|\subsection{text}|. Stejně jako u hlavního nadpisu se čísluje automaticky
   |
    \item \textbf{Itemizace:} Pro itemizaci použijeme prostředí itemize
    
    Jednotlivé body oddělujeme pomocí \verb|\item|.
    \item \textbf{Enumerace:} Principielně funguje stejně jako itemizace, použijeme prostředí \texttt{enumerate}  Jednotlivé body jsou pak místo označení pomlček  očíslovány
    \item U itemizace a enumerace můžeme prostředí do sebe vnořovat

  
    \begin{multicols}{2}
    [ITEMIZACE]
    \setlength{\columnsep}{10cm}
     \begin{lstlisting}[frame = single]
    \begin{itemize}
        \item text1
        \begin{itemize}
            \item text2
        \end{itemize}
    \end{itemize}
    \end{lstlisting}
    \columnbreak
   \begin{centering}
     
   \end{centering}
     \begin{itemize}
        \item text1
        \begin{itemize}
            \item text2
        \end{itemize}
    \end{itemize}
    
    \end{multicols}

          
        
        
    \item Nové odstavce se vytváří automaticky, pokud vynecháme 1 řádek. Řádek jako takový můžeme také zalomit \verb|\\| aniž bychom pak začali nový odstavec viz níže
    \item Pro nezalomitelnou mezeru můžeme použít \verb|~| -- \verb|v~domě|
\end{itemize}
 
    \textbf{Zalamování řádků}
    TEXT TEXT TEXT TEXT TEXT\verb|\\|\\ TEXT TEXT TEXT
    
    \textbf{Vynechání řádku}
    TEXTT TEXT TEXT TEXT TEXT \verb|vynechán řádek|
    
    TEXT TEXT TEXT TEXT
\subsection{Vizuální značky}
\begin{itemize}
    \item \verb|\textbf{text}| pro tučný text - \textbf{text}
    \item \verb|\textit{text}| pro kurzívu - \textit{text}
    \item \verb|\enquote{text}| pro české uvozovky -\enquote{text}
    \item \verb|\texttt{text}| pro strojový text -\texttt{text}

\end{itemize}

\subsection{Obrázky}
\begin{itemize}
    \item Nejdříve do předpřipravené složky (images) v šabloně nahrajeme obrázek, který chceme použít
    \item Obrázky se číslují automaticky
    \item Pro vložení obrázku do textu použijeme prostředí \verb|\begin{figure}...\end{figure}|
    
    \item Určíme\begin{itemize}
        \item Umístění - defaultně vlevo
        \item Cestu k souboru
        \item Caption/popisek obrázku
        \item Label - pomocí labelů můžeme na obrázky odkázat jinde v text pomocí příkladu \verb|\ref{}|
        \item Label a Caption jsou optional body
    \end{itemize}
\end{itemize}
\begin{lstlisting}[frame = single]
    \begin{figure}
        \centering 
        \includegraphics{file_path} 
        \caption{figure caption}
        \label{fig:label}
    \end{figure}
\end{lstlisting}


\begin{subsection}{Matematika a vzorce}
\begin{itemize}
\item In-line mód pomocí prostředí \verb|$ váš vzorec $|
\item Out-line mód pomocí prostředí \verb|$$ váš vzorec $$|
\item Můžeme také použít prostředí \verb|\begin{align}...\end{align}| pokud chceme out-line, očíslovaný vzorce.

\item Důležité symboly:
\begin{itemize}
    
    \item \verb|_| pro lower index \verb|x_1| = $x_1$
    \item \verb|^| pro upper index \verb|x^1| = $x^1$
    \item Pokud chceme do horního nebo spodního indexu napsat vzorec nebo výraz, dáme ho do \verb|{}| -- \verb|x^{e-1}| = $x^{e-1}$
    \item \verb|\{| a \verb|\}| pro složené závorky v math modu.
    \item Pokud chceme zapsat písmeno nebo slovo přímým řezem, použijeme \verb|\mathrm{}| -- používáme při zápisu např. konstant
    \item \verb|\langle| a \verb|\rangle| pro ostré závorky  $\langle \rangle$
    \item Pokud chceme aby se závorka přízpůsobila velikosti vzorce použijeme\\ \verb|\left(......\right)| $$\left(\frac{1}{\frac{1}{2}}\right)$$

    \item \verb|\frac{}{}| pro zlomky \verb|\frac{1}{2}| -- $\frac{1}{2}$
    \item \verb|\sum^x_1{}| pro sumy, pomocí \verb|^| a \verb|_| určíme horní a dolní hranici -- $\sum^1_0$
    \item \verb|\int^1_x{}| pro integrály, pomocí \verb|^| a \verb|_| určíme horní a spodní hranice -- $\int{x+1 \mathrm{ d}x}$
    \item Písmena řecké abecedy: \verb|\anglický_název| -- \verb|\alpha| pro malé alpha, \verb|\Alpha| pro velké Alpha. Stejně se zbytkem písmen.
   
\end{itemize}
\end{itemize}
\begin{multicols}{2}
    \begin{lstlisting}[frame = single]
  E\left\{\chi^2(n)\right\} 
  = E(U_1^2+...+U_n^2) =n
\end{lstlisting}
\columnbreak
$E\left\{\chi^2(n)\right\} = E(U_1^2+...+U_n^2) = n$
\end{multicols}



\end{subsection}

\subsection{Tabulky}
\begin{itemize}
    \item Pro vytváření tabulek použijeme prostředí \verb|\begin{table}{}...\end{table}|
    \item Podobně jako u obrázků můžeme určit parametry:\\ \verb| \centering \caption{} \label{}|
    \item V druhé \verb|{}| určujeme allignement hodnot ve sloupcích - l = left, c = center, r = right. Případně použijeme | pro oddělení daných sloupců lajnou viz příklad níže
    \item Řádek v tabulcu zalamujeme pomocí \verb|\\|
    \item Pro oddělení řádků lajnou použijeme \verb|\hline| za zalomením řádku tabulky, můžeme i několikrát. Opět viz příklad níže.
    \item Jednotlivé celly v tabulce oddělujeme pomocí \verb|&|.
 \end{itemize}
%
%

\begin{multicols}{2}
    [TABULKY]

\begin{lstlisting}[frame = single]
\begin{table}
    \centering
    \begin{tabular}{c|c|c}
    num 1 & num 2 & num 3 \\
    \hline\hline
         1 & 1 & 1  \\ \hline
         2 & 2 & 2  \\ \hline
         3 & 3 & 3  \\ \hline
         4 & 4 & 4  \\ 
    \end{tabular}
\end{table}
\end{lstlisting}
\columnbreak

    \begin{tabular}{c|c|c}
    num 1 & num 2 & num 3 \\ \hline\hline
        1 & 1 & 1  \\ \hline
        2 & 2 & 2  \\ \hline
        3 & 3 & 3  \\ \hline
        4 & 4 & 4  \\ 
    \end{tabular}
\end{multicols}



